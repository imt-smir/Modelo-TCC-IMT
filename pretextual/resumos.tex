% ---
% RESUMOS
% ---

\setlength{\absparsep}{18pt} % ajusta o espaçamento dos parágrafos do resumo
\begin{resumo}
É a apresentação concisa do texto, destacando seus aspectos de maior relevância. Na elaboração do resumo, deve-se:

\begin{itemize}
    \item redigir em um único parágrafo com, no máximo, 500 palavras;
    \item redigir com frases completas e não com sequências de títulos;
    \item usar o verbo na voz ativa e na terceira pessoa do singular;
    \item evitar o uso de citações bibliográficas;
    \item ressaltar os objetivos, metodologia, resultados e conclusões do trabalho;
\end{itemize}

\textbf{Palavras-chaves}: palavras mais representativas do trabalho, separadas entre si por ponto e finalizadas também por ponto.
\end{resumo}

% ABSTRACT in english
\begin{resumo}[Abstract]
 \begin{otherlanguage*}{english}

Resumo em língua extrangeira para divulgação internacional.

   \textbf{Keywords}: X. X. X. X.
 \end{otherlanguage*}
\end{resumo}