\chapter{Testes e Resultados}

Devem ser elaborados testes com o objetivo de validar o funcionamento do que foi desenvolvido. Testes de repetibilidade, precisão, usabilidade, etc. 

Na análise dos resultados são apresentados os resultados obtidos de forma clara e precisa, considerando-se que:

\begin{itemize}
    \item a análise dos dados, sua interpretação e adiscussão teórica podem ser conjugados ou separados, conforme for mais adequado aos objetivos do trabalho;
    \item os diversos resultados obtidos, sem interpretações pessoais, devem vir agrupados e ordenados convenientemente, podendo eventualmente ser acompanhados de tabelas, gráficos, quadros ou figuras para maior clareza;
    \item os dados experimentais obtidos podem ser analisados e relacionados com os principais problemas que existam sobre o assunto, dando subsídios para a conclusão.
\end{itemize}