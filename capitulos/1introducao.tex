% ----------------------------------------------------------
% Introdução 
% Capítulo sem numeração, mas presente no Sumário
% ----------------------------------------------------------

\chapter{Introdução}

Introdução é a parte do trabalho em que o assunto é apresentado em sua totalidade, mas sem detalhes. Trata-se do elemento explicativo do autor para o leitor. A introdução deve:

\begin{itemize}
    \item estabelecer o assunto, definindo-o sucinta e claramente, sem deixar dúvidas quanto ao campo e período abrangidos e incluir informações sobre a natureza e importância do problema;
    \item indicar os objetivos e a finalidade do trabalho, justificar e esclarecer de que ponto de vista é tratado o assunto;
    \item referir-se aos tópicos principais do texto, dando o roteiro ou a ordem de exposição.
    \item referir-se aos tópicos principais do texto, dando o roteiro ou a ordem de exposição.
\end{itemize}

Na introdução não são mencionados os resultados alcançados, pois acarretaria desinteresse pela leitura integral do texto. É recomendável que na introdução estejam descritasa hipóteses, objeto de discussão no trabalho.

\section{Motivações e Justificativas}

\textcolor{red}{Por quê vale a pena pesquisar sobre o assunto?}

\section{Objetivos}

\textcolor{red}{Deve conter uma frase que deixe bem claro o que será desenvolvido.}

``Construção de um robô, controlado por um smartphone, capaz de ...''.

\textcolor{red}{Para facilitar o desenvolvimento do trabalho, pode-se criar objetivos específicos. Além de criar um ``fuxo de trabalho'', estes objetivos específicos auxiliam o leitor a validar se o objetivo principal foi alcançado.}

A fim de alcançar esse objetivo, alguns objetivos específicos foram definidos:
\begin{itemize}
    \item Proposta de um mecanismo para ...;
    \item Desenvolvimento de uma aplicação para ...;
    \item Montagem e testes.
\end{itemize}

\section{Organização do Trabalho}

\textcolor{red}{Como o trabalho está dividido.}

Este trabalho foi dividido em quatro grandes capítulos, com o primeiro sendo a introdução, motivação e objetivos deste trabalho. O segundo capítulo consiste em um estudo sobre... O terceiro capítulo apresenta a montagem do robô... Por fim, os resultados deste trabalho e possíveis melhorias são descritos no quarto capítulo.


